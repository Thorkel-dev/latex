\section{Presentation}
This document is a simple example of a document generated by \gls{latex}. \newline

It is impossible to present all the possible options with \gls{latex} nor those of the packages used
in this document.
There are other ways to do this that you can find on your own.

\subsection{Minimal installation}
Some packages are necessary to generate the document correctly. You can install them easily via your
package manager. Some packages may have different names depending on the manager, here are the
package names for Ubuntu. For example : \mintinline{bash}{apt-get install latexmk}.\newline
The idea here is to install only the packages that will be useful to us rather than installing the
whole package.
\begin{itemize}
    \item latexmk, To automate file compilations
    \item texlive-latex-base, contains packages that are mandated by the core team of \gls{latex}
    \item texlive-latex-extra, contains a very large collection of complementary packages
    \item texlive-font-utils, for image transformation and manipulation
    \item texlive-extra-utils, These are various useful supplementary programmes
    \item texlive-lang-french, French language support
    \item texlive-lang-english, English language support
    \item biber, bibliographic for \gls{latex}
    \item texlive-bibtex-extra, addition of different style for BibLaTeX
    \item python3-pygments, a generic syntax highlighter adapted to code hosting
    \item plantuml, component that allows to draw quickly diagrams
\end{itemize}

\subsubsection{Use in a project}
This is a common template for writing documents. You don't have to modify only the indicated files.

In your git repository, add the git submodule containing the template.
Copy the following files to the directory where you want to write your document:
\begin{itemize}
    \item makefile
    \item latexmkrc
    \item document.tex
    \item tableVersion.tex
    \item acronyms.tex
    \item glossary.tex
    \item biblio.bib
    \item README.md
    \item .gitignore
\end{itemize}

Some files may not be necessary for your document and some may need to be adapted.

\paragraph{Makefile}
The Makefile will recover all the \textbf{.plantuml} to convert them into \textbf{.eps}. See
\ref{plamtuml}, page \pageref{plamtuml} to know how to use it.
Some pathways to files must be adapted to your project.

In the Makefile change the line:
\begin{code}
    \begin{minted}{Makefile}
    SRC_PLANTUML = $(realpath ./schemas)
    to
    SRC_PLANTUML = $(realpath ./<project/path/folder>/schemas)
\end{minted}
    \caption{Makefile change}
\end{code}

\emph{./<project/path/folder>/schemas} is the path to the folder where you will store the various
UML diagrams.\newline

In case this schemas are not needed, you can delete this file, and remove the following line in the
\textbf{latexmkrc} :
\begin{code}
    \begin{minted}{perl}
        # at the start of compilation
        $compiling_cmd = "make eps";
    \end{minted}
    \caption{Delete Makefile}
\end{code}


\paragraph{latexmkrc}
The \textbf{latexmkrc} compiles the file \gls{latex}. See \ref{use}, page \pageref{use} to know how
to use it.
Some pathways to files must be adapted to your project.

In the \textbf{latexmkrc} change the line :
\begin{code}
    \begin{minted}{perl}
    ensure_path("TEXINPUTS", "./param//:", "./figures//");
    to
    ensure_path("TEXINPUTS", "./<project/path/submodule>/param//:", "./<project/path/folder>/submodule/figures//:", "/<project/path/folder>/schemas//:", "/<project/path/folder>/figures//:");
\end{minted}
    \caption{latexmkrc change}
\end{code}

\emph{./<project/path/folder>} is the path to the folders where there are all the Latex source
files, images, ...,.
You can add as many folders as you like as long as they are filled in as above.

\paragraph{document.tex}
This is the central file of your document, the entry point for compilation.
You can rename this file if you wish, but remember to change the following line in the
\textbf{latexmkrc}:

\begin{code}
    \begin{minted}{perl}
        @default_files = ("document.tex");
    \end{minted}
    \caption{Rename file \gls{latex}}
\end{code}

Inside are the parameters given to the \textbf{Parameter} package. See \ref{customization}, page
\pageref{customization} to know how to use it.
The following lines should be replaced by your \gls{latex} files.

\begin{code}
    \begin{minted}{latex}
        %-----------------------------------------
        %            Presentation
        %-----------------------------------------

        % INFO : Add your documentation
        \section{Section}
The purpose here is to present the different titles that exist in the document
and their commands. \newline
\verb=\section{Section}= \newline

\lipsum[1]

\subsection{Subsection}

\verb=\subsection{Subsection}= \newline

\lipsum[2]

\subsubsection{subsubsection}

\verb=\subsubsection{subsubsection}= \newline

\lipsum[3]

\paragraph{Paragraphe}

\verb=\paragraph{Paragraphe}= \newline

\lipsum[4]

\subparagraph{subparagraph}

\verb=\subparagraph{subparagraph}= \newline

\lipsum[5]

\subsubparagraph{subsubparagraph}

\verb=\subsubparagraph{subsubparagraph}= \newline

\lipsum[6]
    \end{minted}
    \caption{Delete document part}
\end{code}

It is recommended that you use multiple files to separate the different parts rather than putting
everything in one file.

\paragraph{tableVersion.tex}
This file contains a table listing the different versions of the documents. See \ref{Xltabular},
page \pageref{Xltabular} to know how to use it.
In this table, add the versions and their information.
This table will be put in the document.\newline

In case this table is not needed, you can delete this file, and remove the following line in the
\textbf{document.tex}:
\begin{code}
    \begin{minted}{latex}
        \section*{Table of versions}
        \begin{xltabular}{\linewidth}{|c|X|>{\centering\arraybackslash}X|c|c|}

    \hline \textbf{Date} & \textbf{Actions} & \textbf{Author} & \textbf{Version} & \textbf{Revisions} \\\hline
    \endfirsthead

    \multicolumn{5}{l}%
    {\textbf{\dots\space continued from previous page}}\\
    \hline \textbf{Date} & \textbf{Actions} & \textbf{Author} & \textbf{Version} & \textbf{Revisions} \\\hline
    \endhead

    \multicolumn{5}{r}%
    {\textbf{Continued on next page\dots}}\tabularnewline
    \endfoot
    \endlastfoot

    20/06/2022 & Creation of the document & Thorkel-dev & 0.0 & 0 \\ \hline
    22/06/2022 & Adding the tables & Thorkel-dev & 0.0 & 1 \\ \hline
    04/07/2022 & Added diagrams & Thorkel-dev & 0.0 & 2 \\ \hline
    17/07/2022 & Addition of the bibliography & Thorkel-dev & 0.0 & 3 \\ \hline
    17/07/2022 & Proofreading and rewriting of the presentation & Thorkel-dev & 0.0 & 4 \\ \hline
    17/07/2022 & First version & Thorkel-dev & 1.0 & 5 \\ \hline
    08/08/2022 & Addition of information & Thorkel-dev & 1.0 & 1 \\ \hline
    08/08/2022 & Front page update & Thorkel-dev & 1.0 & 2 \\ \hline
    08/08/2022 & Version tables & Thorkel-dev & 1.0 & 3 \\ \hline
    08/08/2022 & Second version & Thorkel-dev & 2.0 & 0 \\ \hline
    22/08/2022 & First page translated & Thorkel-dev & 2.0 & 1 \\ \hline
    30/08/2022 & Translation of the entire document & Thorkel-dev & 2.0 & 2 \\ \hline
    02/09/2022 & Change of cover page, header, footer and add landscape & Thorkel-dev & 2.0 & 3 \\ \hline
    08/09/2022 & Add parameters description & Thorkel-dev & 2.0 & 4 \\ \hline
    09/09/2022 & Add background & Thorkel-dev & 2.0 & 5 \\ \hline
    10/09/2022 & Third version & Thorkel-dev & 3.0 & 0 \\ \hline
\end{xltabular} \label{TableOfVersion} % Add the page with the version tables
        \clearpage
    \end{minted}
    \caption{Delete versions table}
\end{code}

\paragraph{acronyms.tex}
This file contains the list of acronyms used in the document. See \ref{acronym}, page
\pageref{acronym} to know how to use it.
This acronyms will be put in the document.\newline

In case this acronyms is not needed, you can delete this file, and remove the following line in the
\textbf{document.tex}:
\begin{code}
    \begin{minted}{latex}
        \loadglsentries{acronyms.tex} % We load the acronyms
        and
        \printglossary[type=\acronymtype, style=superheaderborder] \label{acronyms}
        \glsaddallunused % Even unused acronyms are displayed
    \end{minted}
    \caption{Delete acronyms}
\end{code}

\paragraph{glossary.tex}
This file contains the list of definition used in the document. See \ref{dictionary}, page
\pageref{dictionary} to know how to use it.
This dictionary will be put in the document.\newline

In case this dictionary is not needed, you can delete this file, and remove the following line in
the \textbf{document.tex}:
\begin{code}
    \begin{minted}{latex}
        \loadglsentries{glossary.tex} % We load the dictionary
        and
        \printglossary[] \label{DomainDictionary} % Display the glossary
        \glsaddallunused % Even unused words are displayed
    \end{minted}
    \caption{Delete dictionary}
\end{code}

\paragraph{biblio.bib}
This file contains the list of reference used in the document. See \ref{Biblatex}, page
\pageref{Biblatex} to know how to use it.
This references will be put in the document.\newline

In case this reference is not needed, you can delete this file, and remove the following line in the
\textbf{document.tex}:
\begin{code}
    \begin{minted}{latex}
        \addbibresource{biblio.bib}   % Load the bibliography
        and
        \printbibliography[heading=bibnumbered] \label{bibliography} % Display the bibliography
        \nocite{*} % Even unused references are displayed
    \end{minted}
    \caption{Delete reference}
\end{code}

\subsubsection{Use} \label{use}
To compile the desired document, you must execute,in the directory with the \mintinline{bash}{make}
or the \mintinline{bash}{latexmk}, the command :
\begin{code}
    \begin{minted}{bash}
    make
    or
    latexmk
\end{minted}
    \caption{Generation of document}
\end{code}

If you have deleted the Makefile, the \mintinline{bash}{make} command will not work, use the other
command.\newline

A file with the format \textbf{.pdf} is generated in the \textbf{bin/} folder. Using the
\mintinline{bash}{make} or \mintinline{bash}{latexmk} command gives the same result.
\mintinline{bash}{make} will launch the \mintinline{bash}{latexmk} command which will itself launch
the \mintinline{bash}{make eps} command, see \ref{plamtuml}, page \pageref{plamtuml}.\newline

A \textbf{.temp/} folder is created, it contains all the intermediate compilation files of
\gls{latex}.
A file \textbf{.dependPlantUML} are created in which the compilation dependencies are listed.
They will allow to speed up the next compilations, by compiling only the modified files.

\paragraph{Template customization} \label{customization}
The template uses its own packages for the layout.
They are located in the \textbf{param/} folder.
One is for the front page and the other for the document layout. These two packages have their own
options that will allow you to customize the document.
By filling in the options of \textbf{Parameter} package you also fill in those of
\textbf{FrontPage}. \newline

However, some information must absolutely be filled in. Indeed, they are reused by the packages
afterwards.
\begin{itemize}
    \item \verb=\title{ }=, defines the title of the document
    \item \verb=\date{ }=, defines the date of the document, today's date is set as default\newline
\end{itemize}

Allows you to customize the footer, header, title...
\begin{itemize}
    \item version, optional, define the current version of the document
    \item reference, optional, define the reference of the document
    \item status, optional, define the status of the document
    \item subject, optional, the subject of the document, will be in the document metadata
    \item keywords, optional, the keywords of the document, will be in the document metadata
    \item language, choice of language for culturally determined typographic rules, by default in
          french \label{language}
    \item encoding, the encoding of the document, by default in utf8
    \item projectName,  optional, the name of the document
    \item disclaimer,  optional, path for a file with the disclaimer applicable for the document.
          It could a \gls{latex} file or a text file. It is printed in the front page
    \item sign,  optional, path for the table of signature. the file \textbf{./signature.tex} is
          provided. It is printed in the front page
\end{itemize}

\subsection{Packages}

\begin{itemize}
    \item \href{https://www.ctan.org/pkg/babel}{babel} This package manages the culturally
          determined typographical rules.
    \item \href{https://ctan.org/pkg/fontenc}{fontenc} This package allows displaying and
          to take correctly in charge these accentuated characters (from the point of view of the
          output file).
    \item \href{https://www.ctan.org/pkg/inputenc}{inputenc} This package is used to type the
          accents directly in the source file. It is imperative to specify which encoding is used
          for the source file, here UTF8.
    \item \href{https://www.ctan.org/pkg/inputenc}{xspace} This package provides a single command
          which looks at what comes after it in the that comes after it in the command stream, and
          decides whether to insert a space to replace the one that has been "eaten" by the TeX
          command decoder
    \item \href{https://www.ctan.org/pkg/graphicx}{graphicx} This package relies on
          the graphical package and provides a key-value interface for the optional arguments to the
          \verb=\includegraphics= command. This interface provides functionality that goes far
          beyond what the graphical package offers by itself.
    \item \href{https://www.ctan.org/pkg/lastpage}{lastpage} Reference the number of pages in the
          document via the introduction of a new label.
    \item \href{https://www.ctan.org/pkg/hyperref}{hyperref} This package is used
          to manage cross-referencing commands in \gls{latex} files to produce hypertext links in
          the document.
    \item \href{https://www.ctan.org/pkg/subfiles}{subfiles} This package, allows managing more
          comfortably the projects, by allowing both to process the subsidiary files by themselves
          and to process the main file that which includes them, without making any changes to
          either of them.
    \item \href{https://www.ctan.org/pkg/titlesec}{titlesec} This package provides an interface for
          selecting various title styles.
    \item \href{https://www.ctan.org/pkg/float}{float} Improves the interface for defining floating
          objects such as figures and tables.
    \item \href{https://www.ctan.org/pkg/xcolor}{xcolor} This package starts from the basic
          facilities of the \textbf{color} package, and provides easy, driver-independent access.
          \label{color}
    \item \href{https://www.ctan.org/pkg/fancyhdr}{fancyhdr} The package offers many possibilities,
          both to build headers and footers, and to control their use (for example, when \gls{latex}
          automatically changes the header style used).
    \item \href{https://www.ctan.org/pkg/xltabular}{xltabular} The new xltabular environment is a
          combination of longtable and tabularx : Header and footer definitions, X column specifier,
          and with possible page breaks.
    \item \href{https://www.ctan.org/pkg/enumitem}{enumitem} This package allows the user to control
          the layout of the three basic list environments: enumerate, itemize and description.
    \item \href{https://www.ctan.org/pkg/tocbibind}{tocbibind} Automatically adds the bibliography
          and/or index and/or table of contents, etc., to the table of contents list.
    \item \href{https://www.ctan.org/pkg/geometry}{geometry} The package provides an easy and
          flexible interface for customizing the layout by implementing auto-centering and
          auto-routing mechanisms and self-balancing mechanisms.
    \item \href{https://www.ctan.org/pkg/minted}{minted} The package that facilitates expressive
          syntax colouring in \gls{latex} using the powerful Pygments library
    \item \href{https://www.ctan.org/pkg/caption}{caption} The caption package offers many ways to
          customize captions in floating environments such as figures and tables, and cooperates
          with many other packages.
    \item \href{https://www.ctan.org/pkg/biblatex}{biblatex} BibLaTeX is a complete
          re-implementation of the bibliographic functions provided by \gls{latex}. BibLaTeX uses
          its own data management program called biber to read and process bibliographic data.
          Thanks to biber, BibLaTeX has many features that rival or surpass other bibliographic
          systems.
    \item \href{https://ctan.org/pkg/pdflscape}{pdflscape} The package adds PDF support to the
          landscape environment of package lscape, by setting the PDF /Rotate page attribute. Pages
          with this attribute will be displayed in landscape orientation by conforming PDF viewers.
    \item \href{https://ctan.org/pkg/csquotes}{csquotes} The package provides advanced facilities
          for inline and display quotations.
\end{itemize}

\subsubsection{Dictionary} \label{dictionary}
It is possible to create a dictionary that can be reused in the document.
To do this, you must enter its definitions in the file \textbf{glossary.tex} in the following way:
\begin{code}
    \begin{minted}{latex}
        \newglossaryentry{label}
        {
            name={The word to define},      % Mandatory
            description={The definition},   % Mandatory
            first={The definition (The word)},
            text={The word},
            short={short form},
            plural={plural form},
            symbol={Symbol of the word}
        }

        % For example
        \newglossaryentry{ed}
        {
            name={Example Dictionary},
            description={Term defining the word},
            first={Example Dictionary, first time},
            text={example dictionary},
            short={example},
            plural={examples dictionaries},
            symbol={ED}
        }
\end{minted}
    \caption{Definition in the dictionary}
\end{code}

If a word is used several times in the document and the spelling of this word must be modified
this can be done easily and throughout the document by changing it in \textbf{glossary.tex}.
To call this word, we use the same way as for acronyms, \ref{acronym} page \pageref{acronym}, :
\begin{code}
    \begin{minted}{latex}
        \gls{ed}
\end{minted}
    \caption{Use of the dictionary}
\end{code}

If it is the first time that the word is called it is the definition in the option
\mintinline{latex}{first} that appears: \gls{ed}. A footnote also appears with the definition.
\newline
If it's not the first time: \gls{ed}. It is the word in the \mintinline{latex}{text} option that
appears and there is no footnote.
If the \mintinline{latex}{first} option is not filled, the \mintinline{latex}{name} text is
displayed.

It is possible to display the list of definitions, only the name and description appear. Here it is
at the end of the document, page \pageref{domaindictionary}, sorted alphabetically.

There are other options for the glossary, see
\href{https://fr.overleaf.com/learn/latex/Glossaries}{Overleaf}.

\paragraph{All options}

\begin{code}
    \begin{minted}{latex}
    \gls{<label>}
    To print the term, lowercase. For example, \gls{ed} prints example dictionary when used.

    \Gls{<label>}
    The same as \gls but the first letter will be printed in uppercase. Example: \Gls{ed} prints Example dictionary

    \glspl{<label>}
    The same as \gls but the term is put in its plural form. For instance, \glspl{ed} will write examples dictionaries in your final document.

    \Glspl{<label>}
    The same as \Gls but the term is put in its plural form. For example, \Glspl{ed} renders as Examples dictionaries.

    \glsentryname{<label>}
    Displays the term without a link to the glossary or footnote. Only the name. To be used especially in titles and captions. \glsentryname{ed} prints Example Dictionary

    \glslink{<label>}{<alternate text>}
    This command creates the link as usual, but typesets the alternate text instead. It can also take several options which changes its default behavior (see the documentation).

    \glssymbol{<label>}
    This command prints what ever is defined in symbol=, \glssymbol{ed} renders as ED.

    \glsdesc{<label>}
    This command prints what ever is defined in description=, \glsdesc{ed} will write Term defining the word.
    \end{minted}
    \caption{All options}
\end{code}

\subsubsection{Acronym} \label{acronym}
It is possible to create a list of acronyms that can be reused in the document.
To do this, you must enter your acronyms in the file \textbf{acronyms.tex} in the following way:
\begin{code}
    \inputminted{latex}{./acronyms.tex}
    \caption{Definition of an acronym}
\end{code}

If an acronym is used several times in the document and the spelling of this acronym must be changed
this can be done easily and throughout the document by modifying it in \textbf{acronyms.tex}.
To call this acronym, we use the same way as for the dictionary, \ref{dictionary} page
\pageref{dictionary} :
\begin{code}
    \begin{minted}{latex}
        \gls{acr}
\end{minted}
    \caption{Use of an acronym}
\end{code}

If this is the first time the acronym is called, the definition in the \mintinline{latex}{first}
option appears: \gls{acr}.
A footnote also appears. \newline
If it's not the first time: \gls{acr}. This is the acronym that appears and there is no footnote.
\newline

It is possible to display the list of defined acronyms. Here, page \pageref{acronyms}, a table
sorted alphabetically.

There are other options for acronyms, see
\href{https://fr.overleaf.com/learn/latex/Glossaries}{Overleaf}.

\paragraph{All options}

\begin{code}
    \begin{minted}{latex}
    \acrlong{<label>}
    Displays the phrase which the acronyms stands for. Put the label of the acronym inside the braces. In the example, \acrlong{gcd} prints Greatest Common Divisor.

    \acrshort{<label>} or \gls{<label>}
    Prints the acronym whose label is passed as parameter. For instance, \acrshort{acr} renders as ACR.

    \acrfull{<label>}
    Prints both, the acronym and its definition. In the example the output of \acrfull{acr} is Definition of the acronym (ACR). Different of first option
    \end{minted}
    \caption{All options}
\end{code}

\subsubsection{PlantUML} \label{plamtuml}
For the generation of PlantUML in images it is necessary to have installed:
\begin{itemize}
    \item PlantUML, program allowing to draw UML diagrams, using a simple textual description
    \item Graphviz, generates a graph from a textual description of PlantUML
\end{itemize}

To generate the set of diagrams you need to issue the command :
\begin{code}
    \begin{minted}{bash}
    make eps
\end{minted}
    \caption{Generation of diagrams}
\end{code}

To delete all the files generated by the diagrams:
\begin{code}
    \begin{minted}{bash}
    make clean_eps
\end{minted}
    \caption{Cleaning diagrams}
\end{code}

The command will recover all the \textbf{.plantuml} files
in the \textbf{schemas/} folder to convert them into \textbf{.eps}.
It is a vector image format supported by \gls{latex},
guaranteeing a great image quality whatever the size.

The files are of the following form:
\begin{code}
    \inputminted{vim}{schemas/example.plantuml}
    \caption{Example PlantUML file}
\end{code}

\begin{figure}[H]
    \centering
    \includegraphics[width=2cm]{./schemas/example}
    \caption{Example UML diagram}
\end{figure}

It is possible to store the diagrams in subfolder whose relative path must be specified.
\begin{figure}[H]
    \centering
    \includegraphics[width=2cm]{./schemas/example/exampleFolder}
    \caption{Example UML diagram in a subfolder}
\end{figure}

\subsubsection{Minted}
It is possible to put code fragments in the document via the Minted package.
This package uses the Python library \textbf{Pygments}.
This library must be installed via the package \textbf{python3-pygments}.

The code snippets will be colourized according to the language set as parameter as below:

\begin{code}
    \begin{minted}{c}
    for(int i = 0; i < 10; i++) {
        println("Hello World\n");
    }
\end{minted}
    \caption{Hello World en C}
\end{code}

\begin{code}
    \begin{minted}{java}
    class HelloWorld {
        public static void main(String[] args) {
            System.out.println("Hello, World!");
        }
    }
\end{minted}
    \caption{Hello World en Java}
\end{code}

The code extracts are put in an environment created for the occasion: \mintinline{latex}{code}.
The aim is to add legends and a particular layout and to list them at the end of the document,
page \pageref{TableOfCode}.
It is possible to add a source code file directly into the document, via the command :
\begin{code}
    \begin{minted}{latex}
        \inputminted{python}{hello.py}
\end{minted}
    \caption{Example of integration of source code file}
\end{code}

\subsubsection{Xltabular}
It is possible to create tables so that they can be spread over the whole page and over several
pages. The objective is to place the table on several pages without splitting a line in two.
Here is an example of a table with 5 columns:
\begin{code}
    \begin{minted}{latex}
        \begin{xltabular}{\linewidth}{|l|X|r|c|p{0.15\linewidth}|}
            % l = left, X = remaining space and line break, r = right, c = center, p = space
            % If no size is given to the columns, it is done according to the text in the column

            % Front page title
            \hline \textbf{Left} & \textbf{Back to Line} & \textbf{Right} & \textbf{Center} & \textbf{Fixed Size}\\\hline
            \endfirsthead

            \multicolumn{5}{l}% Merging of columns and text on the left
            {\textbf{\dots\space continued from previous page}}\\
            % Titre seconde page et plus
            \hline \textbf{Left} & \textbf{Back to Line} & \textbf{Right} & \textbf{Center} & \textbf{Fixed Size}\\\hline
            \endhead

            \multicolumn{5}{r}% Merging of columns and text on the right
            {\textbf{Continued on next page\dots}}\tabularnewline
            \endfoot
            \endlastfoot

            Left & Back to Line & Right & Center & Fixed Size\\ \hline
        \end{xltabular}
\end{minted}
    \caption{Example table}
\end{code}

This will give the result:
\begin{xltabular}{\linewidth}{|l|X|r|c|p{0.15\linewidth}|}
    % l = left, X = remaining space and line break, r = right, c = center, p = space
    % If no size is given to the columns, it is done according to the text in the column

    % Front page title
    \hline \textbf{Left} & \textbf{Back to Line} & \textbf{Right} & \textbf{Center} &
    \textbf{Fixed Size}\\\hline
    \endfirsthead

    \multicolumn{5}{l}% Merging of columns and text on the left
    {\textbf{\dots\space continued from previous page}}\\
    % Titre seconde page et plus
    \hline \textbf{Left} & \textbf{Back to Line} & \textbf{Right} & \textbf{Center} &
    \textbf{Fixed Size}\\\hline
    \endhead

    \multicolumn{5}{r}% Merging of columns and text on the right
    {\textbf{Continued on next page\dots}}\tabularnewline
    \endfoot
    \endlastfoot

    Left & Back to Line & Right & Center & Fixed Size\\ \hline
    Left & \lipsum[1] & Right & Center & Fixed Size\\ \hline
    Left & \lipsum[3] & Right & Center & Fixed Size\\ \hline
    Left & \lipsum[5] & Right & Center & Fixed Size\\ \hline
\end{xltabular}

If no size is given in the columns \textbf{Left Right Center}, the width of the column
will be made according to the text that is put in the column. There is a risk of having
tables that are too wide and that protrude from the page like here :
\begin{xltabular}{\linewidth}{|l|X|r|c|p{0.15\linewidth}|}
    % l = left, X = remaining space and line break, r = right, c = center, p = space
    % If no size is given to the columns, it is done according to the text in the column

    % Front page title
    \hline \textbf{Left} & \textbf{Back to Line} & \textbf{Right} & \textbf{Center} &
    \textbf{Fixed Size}\\\hline
    \endfirsthead

    \multicolumn{5}{l}% Merging of columns and text on the left
    {\textbf{\dots\space continued on next page}}\\
    % Title second page and more
    \hline \textbf{Left} & \textbf{Back to Line} & \textbf{Right} & \textbf{Center} &
    \textbf{Fixed Size}\\\hline
    \endhead

    \multicolumn{5}{r}% Merging of columns and text on the right
    {\textbf{Continued on next page\dots}}\tabularnewline
    \endfoot
    \endlastfoot

    \lipsum[1] & Back to Line & Right & Center & Fixed Size\\ \hline
\end{xltabular}

Except in certain cases, see page \pageref{TableOfVersion}, it is preferable to fix everything as
here:
\begin{code}
    \begin{minted}{latex}
        \begin{xltabular}{\linewidth}{|p{0.6\linewidth}|X|}
            % The column is 0.6 times the width of a line
            \hline \textbf{Column 1} & \textbf{Column 2} \\\hline
            \endfirsthead

            \multicolumn{2}{l}%
            {\textbf{\dots\space continued on next page}}\\
            \hline \textbf{Column 1} & \textbf{Column 2} \\\hline
            \endhead

            \multicolumn{2}{r}% Merging of columns and text on the right
            {\textbf{Continued on next page\dots}}\tabularnewline
            \endfoot
            \endlastfoot

            \lipsum[1] & \lipsum[2] \\ \hline
        \end{xltabular}
\end{minted}
    \caption{Example of a fixed size table}
\end{code}

\begin{xltabular}{\linewidth}{|p{0.6\linewidth}|X|}
    % The column is 0.6 times the width of a line
    \hline \textbf{Column 1} & \textbf{Column 2} \\\hline
    \endfirsthead

    \multicolumn{2}{l}%
    {\textbf{\dots\space continued on next page}}\\
    \hline \textbf{Column 1} & \textbf{Column 2} \\\hline
    \endhead

    \multicolumn{2}{r}% Merging of columns and text on the right
    {\textbf{Continued on next page\dots}}\tabularnewline
    \endfoot
    \endlastfoot

    \lipsum[1] & \lipsum[2] \\ \hline
\end{xltabular}

This way, overflow is avoided. Note that if a column is not wide enough and the text inside is more
than one page long more than one-page length, it also overflows.

\subsubsection{Biblatex}
It is possible to create a bibliography that can be reused in the document.
To do this, you must enter the references in the file \textbf{biblio.bib} in the following way:
\begin{code}
    \begin{minted}{latex}
        @article{key to call,
            author = {author},
            title  = {Title},
            ...
        }

        % For example
        @article{CitekeyArticle,
            author  = {P. J. Cohen},
            title   = {Independence of the continuum hypothesis},
            journal = {Proceedings of the National Academy of Sciences},
            year    = 1963,
            volume  = {50},
            number  = {6},
            pages   = {1143--1148}
        }
\end{minted}
    \caption{Definition in the bibliography}
\end{code}

If a reference, is used several times in the document and the spelling of this reference
needs to be changed; this can be done easily and throughout the document by modifying it in
\textbf{biblio.bib}.
To call this reference can use different ways:
\begin{code}
    \begin{minted}{latex}
        \autocite{CitekeyArticle} % Adds the reference as a footnote
        \cite{CitekeyArticle}     % Adds the reference to the suite
\end{minted}
    \caption{Use of the bibliography}
\end{code}

This gives the following result where we add the footnote \autocite{CitekeyArticle}.
And here we have it appearing: \cite{CitekeyArticle}. \newline

In our example we display the author's name and the title. There are other ways to display other
information. See
\href{https://mirror.ibcp.fr/pub/CTAN/macros/latex/contrib/biblatex/doc/biblatex.pdf#subsection.3.9}{documentation CTAN},
or \href{https://fr.overleaf.com/learn/latex/Bibliography_management_with_bibtex}{Overleaf}.
\newline

In the file \textbf{biblio.bib} you will find, as an example, the set of reference categories that
exists. Depending on the category, the information requested is different, and their display is also
different, see page \pageref{bibliography}.

\subsubsection{Csquote}
We may need to quote some excerpts from a document. Here is an example of a quote.
\begin{displayquote}[\cite{CitekeyArticle}]
    It is now known that the truth or falsity of the continuum hypothesis
    and other related conjecture cannot be determined by set theory as we know it today.
\end{displayquote}

For this we use the displayquote environment. This one allows to have a particular layout.
By adding the biblatex reference as an option, we can make the link with the references.
\begin{code}
    \begin{minted}{latex}
        \begin{displayquote}[\cite{CitekeyArticle}]
            It is now known that the truth or falsity of the continuum hypothesis
            and other related conjecture cannot be determined by set theory as we know it today.
        \end{displayquote}
\end{minted}
    \caption{Use of displayquote}
\end{code}

Otherwise, we can simply add the quote in the text as here. \enquote{It is now known that the truth
    or falsity of the continuum hypothesis and other related conjecture cannot be determined by set
    theory as we know it today.} \autocite{CitekeyArticle}
\begin{code}
    \begin{minted}{latex}
\enquote{It is now known that the truth or falsity of the continuum hypothesis and other related
conjecture cannot be determined by set theory as we know it today.}\autocite{CitekeyArticle}
\end{minted}
    \caption{Use of enquote}
\end{code}

It is interesting to note that this package manages quotes within quotes. And when depending on the
language chosen with the \mintinline{latex}{language} parameter, see \ref{language} page
\pageref{language} the quotation marks will not be the same. They will follow the typography rules
of the language.

\enquote{Quotes \enquote{quotes within quotes}}

\begin{landscape}
    \subsubsection{Landscape}
    Depending on the needs, it may be necessary to have a longer line length. In particular to
    display images.
    It is very easy to do thanks to the package pdflscape which offers the landscape environment.
    Environment where you can put what you want as with a normal page
    As here for example:
    \begin{code}
        \begin{minted}{latex}
            \begin{landscape}
                Texts, images, tables, etc.
            \end{landscape}
    \end{minted}
        \caption{Use of landscape}
    \end{code}
\end{landscape}

