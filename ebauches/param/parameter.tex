\usepackage[french]{babel}
\usepackage[T1]{fontenc}
\usepackage[utf8]{inputenc}
\usepackage[pdftex]{graphicx}
\usepackage{xspace}
\usepackage{lastpage}
\usepackage{hyperref}
\usepackage{subfiles}
\usepackage{longtable}
\usepackage{titlesec}
\usepackage{float}
\usepackage[dvipsnames]{xcolor}
\usepackage{fancyhdr}
\usepackage{xltabular}
\usepackage{enumitem}
\usepackage[nottoc,numbib]{tocbibind}
\usepackage[outputdir=build, newfloat]{minted}
\usepackage{caption}

\usepackage{lipsum}
\usepackage[outdir=build/, update]{epstopdf}

\newcommand{\sectionbreak}{\clearpage}

%-----------------------------------------------
%   Personnalisation des marges
%-----------------------------------------------

\usepackage[left=2.5cm,
    right=2.5cm,
    top=2.5cm,
    bottom=2.5cm
]{geometry}
\geometry{body={160mm,230mm},headheight=42pt,top=3cm}
\RequirePackage{kvoptions}

%-----------------------------------------------
%   Variables
%-----------------------------------------------

\graphicspath{{../schemas/}{../figures/}}

\hypersetup{
    colorlinks  = true,
    linkcolor   = black,
    filecolor   = red,
    urlcolor    = blue,
    pdfborder   = 0 0 0
}

\definecolor{ESEORed}{RGB}{59,13,32}
\definecolor{ESEOOrange}{RGB}{91,29,15}
\definecolor{ESEOLightBlue}{RGB}{41,78,87}
\definecolor{ESEODarkBlue}{RGB}{11,46,74}
\definecolor{ESEOGrey}{RGB}{35,35,35}
\definecolor{ESEOWhite}{RGB}{1,1,1}

% itemize personnalisation
\setlist[itemize]{label=\textcolor{black}{\textbullet}}

%-----------------------------------------------
%   Paramètres pour l'en-tête et le pied de page
%-----------------------------------------------

\pagenumbering{arabic}
\pagestyle{fancy}
\makeatletter

% header
\renewcommand{\headrulewidth}{0.4mm}
\fancyhead[LO]{\includegraphics[height= 15pt]{eseo-logo.png} \\ \medskip {\projectName}}
\fancyhead[CO]{\includegraphics[height= 15pt]{thorkel-logo.png} \\ \medskip {\prose} équipe {\teamNumber} {\annee}}
\fancyhead[RO]{\includegraphics[height= 15pt]{latex.png} \\ \medskip {\small{Ref. {\documentNameAbrev}\_E{\teamNumber}}}}
\def\headrule{{\if@fancyplain\let\headrulewidth\plainheadrulewidth\fi
            \color{MidnightBlue}\hrule\@height\headrulewidth\@width\headwidth \vskip-\headrulewidth}}

% footer
\renewcommand{\footrulewidth}{0.4mm}
\fancyfoot[LO]{\textit{Version {\version} - Révision {\revision}}}
\fancyfoot[CO]{\copyright {\annee} E. Gautier, {\teamName}}
\fancyfoot[RO]{Page \thepage /{\pageref{Lastpage}} }
\def\footrule{{\if@fancyplain\let\footrulewidth\plainfootrulewidth\fi
            \vskip-\footruleskip\vskip-\footrulewidth
            \color{MidnightBlue}\hrule\@width\headwidth\@height\footrulewidth\vskip\footruleskip}}

%-----------------------------------------------
%   Les titres des parties
%-----------------------------------------------

% Ajoute un niveau de sous-sous-paragraphe
\newcounter{subsubparagraph}[subparagraph]
\renewcommand\thesubsubparagraph{%
    \thesubparagraph.\@arabic\c@subsubparagraph}
\newcommand\subsubparagraph{%
    \@startsection{subsubparagraph}               % compteur
    {6}                                           % niveau
    {0em}                                         % indentation
    {1em}                                         % avant le titre
    {1em}                                         % après le titre
    {\normalsize\hspace{6em}\color{MidnightBlue}} % style (surcharge par le format du titre)
}
\newcommand\l@subsubparagraph{\@dottedtocline{6}{13em}{6em}}
\newcommand{\subsubparagraphmark}[1]{}
\providecommand*{\toclevel@subsubparagraph}{6}
\makeatother % est à nouveau utilisé normalement

\setcounter{tocdepth}{6}    % Autorise l'alinéa dans la table des matières
\setcounter{secnumdepth}{6} % Permet la numérotation du sous-paragraphe

% format des titres
\titleformat{\section}[block]{\normalfont\bfseries\LARGE\color{MidnightBlue}}{{\thesubsection}}{1em}{}{}
\titleformat{\subsection}[block]{\Large\hspace{2em}\Large\color{MidnightBlue}}{{\thesubsection}}{1em}{}{}
\titleformat{\subsubsection}[block]{\hspace{3em}\large\color{MidnightBlue}}{{\thesubsubsection}}{1em}{}{}
\titleformat{\paragraph}[block]{\normalsize\bfseries\hspace{4em}\color{MidnightBlue}}{{\theparagraph}}{1em}{}{}
\titleformat{\subparagraph}[block]{\normalsize\hspace{5em}\color{MidnightBlue}}{{\thesubparagraph}}{1em}{}{}

%-----------------------------------------------
%   Personnalisation de la liste des figures
%-----------------------------------------------

\renewcommand{\listoffigures}{\begingroup
    \tocsection
    \tocfile{\listfigurename}{lof}
    \endgroup}

\newcounter{@secnumdepth}
\let\oldsection\section
\RenewDocumentCommand{\section}{s o m}{%
    \IfBooleanTF{#1}
    {%
        \setcounter{@secnumdepth}{\value{secnumdepth}}  % Stocke secnumdepth
        \setcounter{secnumdepth}{-1}                    % Afficher seulement jusqu'aux numéros d'ordre \part
        \phantomsection%
        \oldsection{#3}% \section*
        \setcounter{secnumdepth}{\value{@secnumdepth}}  % Restaure secnumdepth
    }
    {%
        \IfValueTF{#2}% \section
        {% \section[.]{..}
            \oldsection[#2]{#3}%
        }
        {% \section{..}
            \oldsection{#3}%
        }
    }
}

%-----------------------------------------------
%   Personnalisation du glossaire
%-----------------------------------------------

\usepackage[acronym, numberedsection, nonumberlist, automake]{glossaries-extra} % Pour faire un glossaire

\glstocfalse
\glsenablehyper
\setglossarystyle{altlistgroup}
\setabbreviationstyle{short-postfootnote}

\ProvideDocumentCommand{\printglossdef}{o m}{\footnote{#1: #2}}
\GlsXtrEnableEntryUnitCounting{general}{0}{page} % Compter le nombre d'occurrences par unité, par page du groupe général
\GlsXtrEnableEntryUnitCounting{abbreviation}{0}{page} % Compter le nombre d'occurrences par unité, par page du groupe abréviation

% S'il s'agit de la première occurrence de l'étiquette
%   alors supprimer l'hyperlien vers le glossaire
% sinon
%   Si l'étiquette est déjà présente dans la page (voir GlsXtrEnableEntryUnitCounting)
%       alors supprimer l'hyperlien
%   sinon ne rien faire
\renewcommand{\glslinkcheckfirsthyperhook}{%
    \glsxtrifwasfirstuse%
    {\setkeys{glslink}{hyper=false}}%
    {%
        \ifnum\glsentrycurrcount\glslabel>0%
            \setkeys{glslink}{hyper=false}%
        \fi
    }%
}

% S'il s'agit de la première occurrence de l'étiquette
%   alors une note de bas de page est ajoutée avec la description liée à l'étiquette
% sinon ne rien faire
\renewcommand{\glsxtrpostlink}{%
    \glsxtrifwasfirstuse%
    {\printglossdef[\glsentryname{\glslabel}]{\glsentrydesc{\glslabel}}}%
    {}%
}

%-----------------------------------------------
%   Minted personnalisation
%-----------------------------------------------

\definecolor{vs}{rgb}{0.95,0.95,0.95}
\setminted{
    linenos=true,
    autogobble=true,         % Aligne avec la marge
    numbers=left,            % Numéro de ligne
    fontsize=\footnotesize,  % Taille de la police
    bgcolor=vs,              % Couleur de fond
    tabsize=4,               % Taille des tabulations
    fontfamily=courier,      % Police d'écriture
    breaklines,              % Retour à la ligne auto
}
% Environnement pour lister le code
\newenvironment{code}{\captionsetup{type=listing}}{\par\hspace{2em}}

\SetupFloatingEnvironment{listing}{name=Source Code}      % Titre dans la légende
\renewcommand*{\listlistingname}{Table des codes sources} % Titre de la liste

%-----------------------------------------------
% Personnalisation de la bibliographie
%-----------------------------------------------

\usepackage[
    backend=biber,
    citestyle=verbose-note, % Style des citations
    style=reading,          % Style de la bibliographie
    entrykey = false,       % On n'affiche pas la clé
    sorting = nty,          % Trie par name, title, year
    hyperref = false,       % Lien entre la citation et la bibliographie
    backref = false         % On n'affiche pas la page des citations
]{biblatex}