\subsection{Installation minimale}
Certains pakage sont nécessaires pour générer correctement le document.
Il s'agit ici d'installer seulement les packages nécessaires.
\begin{itemize}
    \item latexmk, Pour automatiser les compilations du fichier
    \item texlive-latex-base, contient des paquets qui sont mandatés par l'équipe principale de \gls{latex}
    \item texlive-latex-extra, contient une très grande collection de paquets complémentaires
    \item texlive-font-utils, pour la transformation et la manipulation d'image
    \item texlive-lang-french, prise en charge du français
\end{itemize}

\subsection{Package}

\begin{itemize}
    \item \href{https://www.ctan.org/pkg/babel}{babel} Ce paquetage gère les règles
          typographiques culturellement déterminées. Ici en français.
    \item \href{https://ctan.org/pkg/fontenc}{fontenc} Ce paquetage permet quant à lui d’afficher et de
          prendre correctement en charge ces caractères accentués (du point de vue du fichier de sortie).
    \item \href{https://www.ctan.org/pkg/inputenc}{inputenc} Ce paquetage sert à pouvoir taper les accents
          directement dans le fichier source. il est impératif de préciser quel encodage est utilisé pour le fichier
          source, ici UTF8.
    \item \href{https://www.ctan.org/pkg/inputenc}{xspace} Ce paquetage fournit une commande unique qui regarde ce
          qui vient après elle dans le flux de commandes, et décide s'il faut insérer un espace pour remplacer
          celui qui a été "mangé" par le décodeur de commandes TeX
    \item \href{https://www.ctan.org/pkg/graphicx}{graphicx} Ce paquetage s'appuie sur
          le paquetage graphique et fournit une interface clé-valeur pour les arguments
          facultatifs de la commande \verb=\includegraphics=. Cette interface fournit des
          fonctionnalités qui vont bien au-delà de ce que le paquetage graphique offre par lui-même.
    \item \href{https://www.ctan.org/pkg/lastpage}{lastpage} Référencez le nombre de pages de du
          document \gls{latex} par l'intermédiaire de l'
          l'introduction d'une nouvelle étiquette.
    \item \href{https://www.ctan.org/pkg/hyperref}{hyperref} Ce paquetage est utilisé
          pour gérer les commandes de référencement croisé dans les fichiers
          \gls{latex} pour produire des liens hypertextes dans le document.
    \item \href{https://www.ctan.org/pkg/subfiles}{subfiles} Ce paquetage, permet de gérer plus confortablement les projets
          multi-fichiers, en permettant à la fois de traiter les fichiers subsidiaires par eux-mêmes et de traiter le
          fichier principal qui les inclut, sans apporter de modifications à l'un ou l'autre.
    \item \href{https://www.ctan.org/pkg/titlesec}{titlesec} Ce paquetage fournit une interface pour la sélection de
          divers styles de titres.
    \item \href{https://www.ctan.org/pkg/float}{float} Améliore l'interface de définition des objets flottants
          tels que les figures et les tableaux.
    \item \href{https://www.ctan.org/pkg/xcolor}{xcolor} Ce paquetage part des installations de base
          du paquetage \textbf{color}, et fournit un accès facile et indépendant du pilote.
    \item \href{https://www.ctan.org/pkg/fancyhdr}{fancyhdr} Le paquetage offre de nombreuses possibilités,
          à la fois pour construire des en-têtes et des pieds de page, et pour contrôler leur
          utilisation (par exemple, lorsque \gls{latex} change automatiquement le style d'en-tête utilisé).
    \item \href{https://www.ctan.org/pkg/xltabular}{xltabular} Le nouvel environnement xltabular est une combinaison
          de longtable et tabularx : Définitions d'en-tête et de pied de page, spécificateur de colonne X,
          et avec des sauts de page possibles.
    \item \href{https://www.ctan.org/pkg/enumitem}{enumitem} Ce paquetage permet à l'utilisateur de contrôler
          la disposition des trois environnements de liste de base : enumerate, itemize et description.
    \item \href{https://www.ctan.org/pkg/tocbibind}{tocbibind} Ajoute automatiquement la bibliographie et/ou
          l'index et/ou le sommaire, etc., à la liste de la table des matières.
    \item \href{https://www.ctan.org/pkg/geometry}{geometry} Le paquetage fournit une interface
          utilisateur facile et flexible pour personnaliser la mise en page en mettant en œuvre des mécanismes
          d'auto-centrage et d'auto-équilibrage.
    \item \href{https://www.ctan.org/pkg/minted}{minted} Le paquetage qui facilite la coloration syntaxique expressive dans
          LaTeX en utilisant la puissante bibliothèque Pygments
    \item \href{https://www.ctan.org/pkg/caption}{caption} Le paquetage de légendes offre de nombreuses façons de
          personnaliser les légendes dans les environnements flottants comme les figures et les tableaux, et coopère
          avec de nombreux autres paquets.
\end{itemize}

\subsubsection{Dictionnaire} \label{dictionnaire}
Il est possible de créer un dictionnaire qui pourrat être réutilisé dans le document.
Pour cela, il faut renseigner ces définitions dans le fichier \verb=glossary.tex= de la manière suivante :
\begin{code}
    \begin{minted}{latex}
        \newglossaryentry{la clé pour appeller}
        {
        name={Le mot à définir},
        description={La définition},
        first={La définition (Le mot)},
        text={Le mot},
        }
        % Par exemple
        \newglossaryentry{ed}
        {
            name={Exemple Dictionnaire},
            description={Terme définissant le mot},
            first={Exemple Dictionnaire (ED)},
            text={ED},
        }
\end{minted}
    \caption{Définition d'en le dictionnaire}
\end{code}

Si un mot est utilisé plusieurs fois dans le document et que l'orthographe de cet acronyme doit être modifiée;
cela peut se faire facilement et dans tout le document en le modifiant dans \verb=glossary.tex=.
On utilise, pour appeler ce mot, la même manière que pour les acronymes, \ref{acronyme} page \pageref{acronyme}, :
\begin{code}
    \begin{minted}{latex}
        \gls{ed}
\end{minted}
    \caption{Utilisation du dictionnaire}
\end{code}

Si c'est la première fois que le mot est appelé c'est la définition dans l'option \verb=first= qui apparaît : \gls{ed}.
Une note de bas de page apparaît aussi. \newline
Si ça n'est pas la première fois: \gls{ed}. C'est le mot dans l'option \verb=text= qui apparaît et il n'y a pas de note de bas de page.

Il est possible de faire apparaitre la liste des définitions. Ici c'est en fin de document,
page \pageref{dictionnaireDomaine}, trié par ordre alphabétique:

\subsubsection{Acronyme} \label{acronyme}
Il est possible de créer une liste d'acronymes qui pourrat être réutilisé dans le document.
Pour cela, il faut renseigner ces acronymes dans le fichier \verb=acronyms.tex= de la manière suivante :
\begin{code}
    \begin{minted}{latex}
        \newacronym[first=Définition de l'acronyme (l'acronyme)]{la clé pour appeller}{l'acronyme}{Définition de l'acronyme}
        % Par exemple
        \newacronym[first=Définition de l'acronyme (ARC)]{acr}{ARC}{Définition de l'acronyme}
\end{minted}
    \caption{Définition d'un acronyme}
\end{code}

Si un acronyme est utilisé plusieurs fois dans le document et que l'orthographe de cet acronyme doit être modifiée;
cela peut se faire facilement et dans tout le document en le modifiant dans \verb=acronyms.tex=.
On utilise, pour appeler cet acronyme, la même manière que pour le dictionnaire, \ref{dictionnaire} page \pageref{dictionnaire}, :
\begin{code}
    \begin{minted}{latex}
        \gls{acr}
\end{minted}
    \caption{Utilisation d'un acronyme}
\end{code}

Si c'est la première fois que l'acronyme est appelé c'est la définition dans l'option \verb=first= qui apparaît : \gls{acr}.
Une note de bas de page apparaît aussi. \newline
Si ça n'est pas la première fois: \gls{acr}. C'est l'acronyme qui apparaît et il n'y a pas de note de bas de page.

Il est possible de faire apparaitre la liste des acronymes défini. Ici c'est dans un tableau trié par ordre alphabétique:

\setglossarysection{paragraph} % Type de titre
\printglossary[type=\acronymtype,style=superheaderborder ,title={Définitions, acronymes et abréviations}]
\glsaddallunused

\subsubsection{PlantUMLs}
Pour la génération des PlantUMLs en images il est nécessaire d'avoir installé:
\begin{itemize}
    \item Java, utilisation de \verb=plantuml.jar=. Le chemin relatif vers
          \verb=plantuml.jar= doit être indiqué dans le \verb=makefile= via la variable \verb=PLANTUML=.
    \item Graphviz, génère un graphe à partir d'une description textuelle de PlantUML
\end{itemize}

Pour générer l'ensemble des diagrammes faire la commande :
\begin{code}
    \begin{minted}{bash}
    make eps
\end{minted}
    \caption{Génération des diagrammes}
\end{code}

Pour supprimer tout les fichiers générés par les diagrammes :
\begin{code}
    \begin{minted}{bash}
    make clean_eps
\end{minted}
    \caption{Nétoyage des diagrammes}
\end{code}

La commande va récupérer l'ensemble des fichiers \verb=.plantuml=
dans le dossier \verb=schémas/= pour les convertir en \verb=.eps=.
Il s'agit d'un format d'image vectorielle supporté pas \gls{latex}.
Garantissant une grande qualité d'image quelle que soit la taille.

Les fichier sont de la forme suivant :
\begin{code}
    \begin{minted}{vim}
        @startuml ExempleFolder ' Titre du fichier générer
        skinparam shadowing false

        Alice -> Bob : hello

        @enduml ' Fin du document
    \end{minted}
    \caption{Exemple de fichier PlantUMLs}
\end{code}

\begin{figure}[H]
    \centering
    \includegraphics[width=2cm]{Exemple}
    \caption{Exemple de diagramme UML}
\end{figure}

Il est posible de rangers les diagrammes dans des sous-dossiers dont il faudra préciser le chemin relatif.
\begin{figure}[H]
    \centering
    \includegraphics[width=2cm]{exemple/ExempleFolder}
    \caption{Exemple de diagramme UML dans un sous dossier }
\end{figure}

\subsubsection{Minted}
Il est possible de mettre des extraits de code dans le document via le paquetage Minted.
Ce paquetage utilise la librairie Python \verb=Pygments=. Il faut l'installer via la commande :
\begin{code}
    \begin{minted}{bash}
    pip install pygments
\end{minted}
    \caption{Hello World en C}
\end{code}

Les extraits de code seront colorisé en fonction du langage mit en paramètre comme ci-dessous :

\begin{code}
    \begin{minted}{c}
    for(int i = 0; i < 10; i++) {
        println("Hello World\n");
    }
\end{minted}
    \caption{Hello World en C}
\end{code}

\begin{code}
    \begin{minted}{java}
    class HelloWorld {
        public static void main(String[] args) {
            System.out.println("Hello, World!");
        }
    }
\end{minted}
    \caption{Hello World en Java}
\end{code}

Les extraits de code sont mis dans un environnement crée pour location \verb=code=.
Le but etant d'ajouter des légendes et une mise en page particulière et de les listes par la suite à la fin du document,
page \pageref{TableOfCode}.

\subsubsection{xltabular}