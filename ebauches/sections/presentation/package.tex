\subsection{Packages}

\begin{itemize}
    \item \href{https://www.ctan.org/pkg/babel}{babel} This package manages the culturally
          determined typographical rules.
    \item \href{https://ctan.org/pkg/fontenc}{fontenc} This package allows to display and
          to take correctly in charge these accentuated characters (from the point of view of the output file).
    \item \href{https://www.ctan.org/pkg/inputenc}{inputenc} This package is used to type the accents
          directly in the source file. It is imperative to specify which encoding is used for the
          source file, here UTF8.
    \item \href{https://www.ctan.org/pkg/inputenc}{xspace} This package provides a single command which looks at what comes after it in the
          that comes after it in the command stream, and decides whether to insert a space to replace
          the one that has been "eaten" by the TeX command decoder
    \item \href{https://www.ctan.org/pkg/graphicx}{graphicx} This package relies on
          the graphical package and provides a key-value interface for the optional arguments
          optional arguments to the \verb=\includegraphics= command. This interface provides
          functionality that goes far beyond what the graphical package offers by itself.
    \item \href{https://www.ctan.org/pkg/lastpage}{lastpage} Reference the number of pages in the
          document via the introduction of a new label.
    \item \href{https://www.ctan.org/pkg/hyperref}{hyperref} This package is used
          to manage cross-referencing commands in \gls{latex} files to produce hypertext links in the document.
    \item \href{https://www.ctan.org/pkg/subfiles}{subfiles} This package, allows to manage more comfortably the projects,
          by allowing both to process the subsidiary files by themselves and to process the main file that
          which includes them, without making any changes to either of them.
    \item \href{https://www.ctan.org/pkg/titlesec}{titlesec} This package provides an interface for selecting
          various title styles.
    \item \href{https://www.ctan.org/pkg/float}{float} Improves the interface for defining floating objects
          such as figures and tables.
    \item \href{https://www.ctan.org/pkg/xcolor}{xcolor} This package starts from the basic facilities
          of the \textbf{color} package, and provides easy, driver-independent access.
    \item \href{https://www.ctan.org/pkg/fancyhdr}{fancyhdr} The package offers many possibilities,
          both to build headers and footers, and to control their use (for example, when \gls{latex} automatically changes the header style used).
    \item \href{https://www.ctan.org/pkg/xltabular}{xltabular} The new xltabular environment is a combination
          of longtable and tabularx : Header and footer definitions, X column specifier,
          and with possible page breaks.
    \item \href{https://www.ctan.org/pkg/enumitem}{enumitem} This package allows the user to control
          the layout of the three basic list environments: enumerate, itemize and description.
    \item \href{https://www.ctan.org/pkg/tocbibind}{tocbibind} Automatically adds the bibliography and/or
          index and/or table of contents, etc., to the table of contents list.
    \item \href{https://www.ctan.org/pkg/geometry}{geometry} The package provides an easy and flexible
          interface for customizing the layout by implementing auto-centering and auto-routing mechanisms.
          and self-balancing mechanisms.
    \item \href{https://www.ctan.org/pkg/minted}{minted} The package that facilitates expressive syntax coloring in
          \gls{latex} using the powerful Pygments library
    \item \href{https://www.ctan.org/pkg/caption}{caption} The caption package offers many ways to customize captions
          in floating environments such as figures and tables, and cooperates with with many other packages.
    \item \href{https://www.ctan.org/pkg/biblatex}{biblatex} BibLaTeX is a complete reimplementation of the bibliographic functions
          functions provided by \gls{latex}. BibLaTeX uses its own data management program called biber
          to read and process bibliographic data. Thanks to biber, BibLaTeX has many features that
          that rival or surpass other bibliographic systems.
\end{itemize}

\subsubsection{Dictionary} \label{dictionary}
It is possible to create a dictionary that can be reused in the document.
To do this, you must enter its definitions in the file \verb=glossary.tex= in the following way:
\begin{code}
    \begin{minted}{latex}
        \newglossaryentry{the key to call}
        {
        name={The word to define},
        description={The definition},
        first={The definition (The word)},
        text={The word},
        }

        % For example
        \newglossaryentry{ed}
        {
            name={Example Dictionary},
            description={Word defining term},
            first={Example Dictionary (ED)},
            text={ED},
        }
\end{minted}
    \caption{Definition in the dictionary}
\end{code}

If a word is used several times in the document and the spelling of this word must be modified
this can be done easily and throughout the document by changing it in \verb=glossary.tex=.
To call this word, we use the same way as for acronyms, \ref{acronym} page \pageref{acronym}, :
\begin{code}
    \begin{minted}{latex}
        \gls{ed}
\end{minted}
    \caption{Use of the dictionary}
\end{code}

If it is the first time that the word is called it is the definition in the option \verb=first= that appears: \gls{ed}.
A footnote also appears with the definition. \newline
If it's not the first time: \gls{ed}. It is the word in the \verb=text= option that appears and there is no footnote.

It is possible to display the list of definitions. Here it is at the end of the document,
page \pageref{domaindictionary}, sorted alphabetically.

There are other options for the glossary, see \href{https://fr.overleaf.com/learn/latex/Glossaries}{Overleaf}.

\subsubsection{Acronym} \label{acronym}
It is possible to create a list of acronyms that can be reused in the document.
To do this, you must enter your acronyms in the file \verb=acronyms.tex= in the following way:
\begin{code}
    \inputminted{latex}{./acronyms.tex}
    \caption{Definition of an acronym}
\end{code}

If an acronym is used several times in the document and the spelling of this acronym must be changed
this can be done easily and throughout the document by modifying it in \verb=acronyms.tex=.
To call this acronym, we use the same way as for the dictionary, \ref{dictionary} page \pageref{dictionary} :
\begin{code}
    \begin{minted}{latex}
        \gls{acr}
\end{minted}
    \caption{Use of an acronym}
\end{code}

If this is the first time the acronym is called, the definition in the \verb=first= option appears: \gls{acr}.
A footnote also appears. \newline
If it's not the first time: \gls{acr}. This is the acronym that appears and there is no footnote. \newline

It is possible to display the list of defined acronyms. Here it is in a table sorted alphabetically.

\setglossarysection{paragraph} % Type of title
\printglossary[type=\acronymtype,style=superheaderborder ,title={Definitions, acronyms and abbreviations}]
\glsaddallunused \newline % Even unused acronyms are displayed

There are other options for acronyms, see \href{https://fr.overleaf.com/learn/latex/Glossaries}{Overleaf}.

\subsubsection{PlantUML}
Pour la génération des PlantUMLs en images il est nécessaire d'avoir installé:
\begin{itemize}
    \item PlantUML, programme permettant de dessiner des diagrammes UML, en utilisant une simple description textuelle
    \item Graphviz, génère un graphe à partir d'une description textuelle de PlantUML
\end{itemize}

Pour générer l'ensemble des diagrammes il faut faire la commande :
\begin{code}
    \begin{minted}{bash}
    make eps
\end{minted}
    \caption{Génération des diagrammes}
\end{code}

Pour supprimer tout les fichiers générés par les diagrammes :
\begin{code}
    \begin{minted}{bash}
    make clean_eps
\end{minted}
    \caption{Nettoyage des diagrammes}
\end{code}

La commande va récupérer l'ensemble des fichiers \verb=.plantuml=
dans le dossier \verb=schémas/= pour les convertir en \verb=.eps=.
Il s'agit d'un format d'image vectorielle supporté par \gls{latex},
garantissant une grande qualité d'image quelle que soit la taille.

Les fichier sont de la forme suivant :
\begin{code}
    \inputminted{vim}{../schemas/exemple.plantuml}
    \caption{Exemple de fichier PlantUMLs}
\end{code}

\begin{figure}[H]
    \centering
    \includegraphics[width=2cm]{exemple}
    \caption{Exemple de diagramme UML}
\end{figure}

Il est posible de ranger les diagrammes dans des sous-dossiers dont il faudra préciser le chemin relatif.
\begin{figure}[H]
    \centering
    \includegraphics[width=2cm]{exemple/exempleFolder}
    \caption{Exemple de diagramme UML dans un sous dossier }
\end{figure}

\subsubsection{Minted}
Il est possible de mettre des extraits de code dans le document via le paquetage Minted.
Ce paquetage utilise la librairie Python \verb=Pygments=.
Cette librairie doit être installée via le paquet \verb=python3-pygments=.

Les extraits de code seront colorisés en fonction du langage mit en paramètre comme ci-dessous :

\begin{code}
    \begin{minted}{c}
    for(int i = 0; i < 10; i++) {
        println("Hello World\n");
    }
\end{minted}
    \caption{Hello World en C}
\end{code}

\begin{code}
    \begin{minted}{java}
    class HelloWorld {
        public static void main(String[] args) {
            System.out.println("Hello, World!");
        }
    }
\end{minted}
    \caption{Hello World en Java}
\end{code}

Les extraits de code sont mis dans un environnement crée pour l'occasion : \verb=code=.
Le but étant d'ajouter des légendes et une mise en page particulière et de les lister par la suite à la fin du document,
page \pageref{TableOfCode}.
Il est possible d'ajouter directement un ficher de code source dans le document, via la commande :
\begin{code}
    \begin{minted}{latex}
        \inputminted{python}{hello.py}
\end{minted}
    \caption{Exemple d'intégration du fichier code source}
\end{code}

\subsubsection{Xltabular}
Il est possible de créer des tableaux pour qu'ils puissent s'étaler sur l'ensemble de la page et sur plusieurs pages.
L'objectif de placer le tableau sur plusieurs page sans qu'une ligne ne soit divisée en deux.
Voici un exemple pour un tableau à 5 colonnes:
\begin{code}
    \begin{minted}{latex}
        \begin{xltabular}{\linewidth}{|l|X|r|c|p{0.15\linewidth}|}
            % l = left, X = place restante et retour à la ligne, r = right, c = center, p = place
            % Si aucune taille n'est donnée aux colonnes, elle se fait en fonction du texte dans la colonne

            % Titre première page
            \hline \textbf{Gauche} & \textbf{Retour à la ligne} & \textbf{Droite} & \textbf{Centre} & \textbf{Taille fixe}\\\hline
            \endfirsthead

            \multicolumn{5}{l}% Fusion des colones et texte à gauche
            {\textbf{\dots\space suite de la page précédente}}\\
            % Titre seconde page et plus
            \hline \textbf{Gauche} & \textbf{Retour à la ligne} & \textbf{Droite} & \textbf{Centre} & \textbf{Taille fixe}\\\hline
            \endhead

            \multicolumn{5}{r}% Fusion des colones et texte à droite
            {\textbf{Suite à la page suivante\dots}}\tabularnewline
            \endfoot
            \endlastfoot

            Gauche & Retour à la ligne & Droite & Centre & Taille fixe\\ \hline
        \end{xltabular}
\end{minted}
    \caption{Exemple tableau}
\end{code}

Ce qui donnera le résultat :
\begin{xltabular}{\linewidth}{|l|X|r|c|p{0.15\linewidth}|}
    % l = left, X = place restante et retour à la ligne, r = right, c = center, p = place
    % Si aucune taille n'est donnée aux colonnes, elle se fait en fonction du texte dans la colonne

    % Titre première page
    \hline \textbf{Gauche} & \textbf{Retour à la ligne} & \textbf{Droite} & \textbf{Centre} & \textbf{Taille fixe}\\\hline
    \endfirsthead

    \multicolumn{5}{l}% Fusion des colones et texte à gauche
    {\textbf{\dots\space suite de la page précédente}}\\
    % Titre seconde page et plus
    \hline \textbf{Gauche} & \textbf{Retour à la ligne} & \textbf{Droite} & \textbf{Centre} & \textbf{Taille fixe}\\\hline
    \endhead

    \multicolumn{5}{r}% Fusion des colones et texte à droite
    {\textbf{Suite à la page suivante\dots}}\tabularnewline
    \endfoot
    \endlastfoot

    Gauche & Retour à la ligne & Droite & Centre & Taille fixe\\ \hline
    Gauche & \lipsum[1] & Droite & Centre & Taille fixe\\ \hline
    Gauche & \lipsum[3] & Droite & Centre & Taille fixe\\ \hline
    Gauche & \lipsum[5] & Droite & Centre & Taille fixe\\ \hline
\end{xltabular}

Si aucune taille n'est donnée dans les colones \verb=Gauche Droite Centre=, la largueur de la colonne
se fera en fonction du texte qui est mit dans la colonne. On riques d'avoir des tableaux trop large qui dépassent de la page
comme ici :
\begin{xltabular}{\linewidth}{|l|X|r|c|p{0.15\linewidth}|}
    % l = left, X = place restante et retour à la ligne, r = right, c = center, p = place
    % Si aucune taille n'est donnée aux colonnes, elle se fait en fonction du texte dans la colonne

    % Titre première page
    \hline \textbf{Gauche} & \textbf{Retour à la ligne} & \textbf{Droite} & \textbf{Centre} & \textbf{Taille fixe}\\\hline
    \endfirsthead

    \multicolumn{5}{l}% Fusion des colones et texte à gauche
    {\textbf{\dots\space suite de la page précédente}}\\
    % Titre seconde page et plus
    \hline \textbf{Gauche} & \textbf{Retour à la ligne} & \textbf{Droite} & \textbf{Centre} & \textbf{Taille fixe}\\\hline
    \endhead

    \multicolumn{5}{r}% Fusion des colones et texte à droite
    {\textbf{Suite à la page suivante\dots}}\tabularnewline
    \endfoot
    \endlastfoot

    \lipsum[1] & Retour à la ligne & Droite & Centre & Taille fixe\\ \hline
\end{xltabular}

Sauf dans certain cas, voir page \pageref{TableOfVersion}, il est préférable de tout fixer comme ici :
\begin{code}
    \begin{minted}{latex}
        \begin{xltabular}{\linewidth}{|p{0.6\linewidth}|X|}
            % La colonne fait 0,6 fois largueur d'une ligne
            \hline \textbf{Colonne 1} & \textbf{Colonne 2} \\\hline
            \endfirsthead

            \multicolumn{2}{l}%
            {\textbf{\dots\space suite de la page précédente}}\\
            \hline \textbf{Colonne 1} & \textbf{Colonne 2} \\\hline
            \endhead

            \multicolumn{2}{r}% Fusion des colones et texte à droite
            {\textbf{Suite à la page suivante\dots}}\tabularnewline
            \endfoot
            \endlastfoot

            \lipsum[1] & \lipsum[2] \\ \hline
        \end{xltabular}
\end{minted}
    \caption{Exemple tableau taille fixe}
\end{code}

\begin{xltabular}{\linewidth}{|p{0.6\linewidth}|X|}
    % La colonne fait 0,6 fois largueur d'une ligne
    \hline \textbf{Colonne 1} & \textbf{Colonne 2} \\\hline
    \endfirsthead

    \multicolumn{2}{l}%
    {\textbf{\dots\space suite de la page précédente}}\\
    \hline \textbf{Colonne 1} & \textbf{Colonne 2} \\\hline
    \endhead

    \multicolumn{2}{r}% Fusion des colones et texte à droite
    {\textbf{Suite à la page suivante\dots}}\tabularnewline
    \endfoot
    \endlastfoot

    \lipsum[1] & \lipsum[2] \\ \hline
\end{xltabular}

Ainsi, on évite des débordements. À noter que si une colonne n'est pas assez large et que le texte à l'intérieur fait
plus d'une longueur de page cela déborde aussi.

\subsubsection{Biblatex}
Il est possible de créer une bibliographie qui pourra être réutilisée dans le document.
Pour cela, il faut renseigner ses références dans le fichier \verb=biblio.bib= de la manière suivante :
\begin{code}
    \begin{minted}{latex}
        @article{la clé pour appeller,
            author  = {auteur},
            title   = {Titre},
            ...
        }

        % Par exemple
        @article{CitekeyArticle,
            author  = {P. J. Cohen},
            title   = {The independence of the continuum hypothesis},
            journal = {Proceedings of the National Academy of Sciences},
            year    = 1963,
            volume  = {50},
            number  = {6},
            pages   = {1143--1148}
        }
\end{minted}
    \caption{Définition dans la bibliographie}
\end{code}

Si une référence, est utilisée plusieurs fois dans le document et que l’orthographe de cette référence
doit être modifié ; cela peut se faire facilement et dans tout le document en le modifiant dans
\verb=biblio.bib=.
Pour appeler cette référence peut utiliser différentes manières:
\begin{code}
    \begin{minted}{latex}
        \autocite{CitekeyArticle} % Ajoute la référence en note de bas de page
        \cite{CitekeyArticle}     % Ajoute la référence à la suite
\end{minted}
    \caption{Utilisation de la bibliographie}
\end{code}

Ce qui donne le résultat suivant où l'on ajoute la note de bas de page \autocite{CitekeyArticle}.
Et ici on a la fait apparaitre : \cite{CitekeyArticle}. \newline

Dans notre exemple on affiche le nom de l'auteur et le titre. Il existe d'autres manière où l'on affiche d'autres informations.
Voir la \href{https://mirror.ibcp.fr/pub/CTAN/macros/latex/contrib/biblatex/doc/biblatex.pdf#subsection.3.9}{documantation CTAN}, ou \href{https://fr.overleaf.com/learn/latex/Bibliography_management_with_bibtex}{Overleaf}. \newline

Dans le fichier \verb=biblio.bib= se trouve, en exemple, l'ensemble des catégories de référence qui existe.
En fonction de la catégorie, les informations demandées sont différentes et leurs affichage le sont aussi,
voir page \pageref{bibliographie}.