\subsection{Paquetages}

\begin{itemize}
    \item \href{https://www.ctan.org/pkg/babel}{babel} Ce paquetage gère les règles
          typographiques culturellement déterminées. Ici en français.
    \item \href{https://ctan.org/pkg/fontenc}{fontenc} Ce paquetage permet quant à lui d’afficher et de
          prendre correctement en charge ces caractères accentués (du point de vue du fichier de sortie).
    \item \href{https://www.ctan.org/pkg/inputenc}{inputenc} Ce paquetage sert à pouvoir taper les accents
          directement dans le fichier source. il est impératif de préciser quel encodage est utilisé pour le fichier
          source, ici UTF8.
    \item \href{https://www.ctan.org/pkg/inputenc}{xspace} Ce paquetage fournit une commande unique qui regarde ce
          qui vient après elle dans le flux de commandes, et décide s'il faut insérer un espace pour remplacer
          celui qui a été "mangé" par le décodeur de commandes TeX
    \item \href{https://www.ctan.org/pkg/graphicx}{graphicx} Ce paquetage s'appuie sur
          le paquetage graphique et fournit une interface clé-valeur pour les arguments
          facultatifs de la commande \verb=\includegraphics=. Cette interface fournit des
          fonctionnalités qui vont bien au-delà de ce que le paquetage graphique offre par lui-même.
    \item \href{https://www.ctan.org/pkg/lastpage}{lastpage} Référencez le nombre de pages de du
          document \gls{latex} par l'intermédiaire de l'
          l'introduction d'une nouvelle étiquette.
    \item \href{https://www.ctan.org/pkg/hyperref}{hyperref} Ce paquetage est utilisé
          pour gérer les commandes de référencement croisé dans les fichiers
          \gls{latex} pour produire des liens hypertextes dans le document.
    \item \href{https://www.ctan.org/pkg/subfiles}{subfiles} Ce paquetage, permet de gérer plus confortablement les projets
          multi-fichiers, en permettant à la fois de traiter les fichiers subsidiaires par eux-mêmes et de traiter le
          fichier principal qui les inclut, sans apporter de modifications à l'un ou l'autre.
    \item \href{https://www.ctan.org/pkg/titlesec}{titlesec} Ce paquetage fournit une interface pour la sélection de
          divers styles de titres.
    \item \href{https://www.ctan.org/pkg/float}{float} Améliore l'interface de définition des objets flottants
          tels que les figures et les tableaux.
    \item \href{https://www.ctan.org/pkg/xcolor}{xcolor} Ce paquetage part des installations de base
          du paquetage \textbf{color}, et fournit un accès facile et indépendant du pilote.
    \item \href{https://www.ctan.org/pkg/fancyhdr}{fancyhdr} Le paquetage offre de nombreuses possibilités,
          à la fois pour construire des en-têtes et des pieds de page, et pour contrôler leur
          utilisation (par exemple, lorsque \gls{latex} change automatiquement le style d'en-tête utilisé).
    \item \href{https://www.ctan.org/pkg/xltabular}{xltabular} Le nouvel environnement xltabular est une combinaison
          de longtable et tabularx : Définitions d'en-tête et de pied de page, spécificateur de colonne X,
          et avec des sauts de page possibles.
    \item \href{https://www.ctan.org/pkg/enumitem}{enumitem} Ce paquetage permet à l'utilisateur de contrôler
          la disposition des trois environnements de liste de base : enumerate, itemize et description.
    \item \href{https://www.ctan.org/pkg/tocbibind}{tocbibind} Ajoute automatiquement la bibliographie et/ou
          l'index et/ou le sommaire, etc., à la liste de la table des matières.
    \item \href{https://www.ctan.org/pkg/geometry}{geometry} Le paquetage fournit une interface
          utilisateur facile et flexible pour personnaliser la mise en page en mettant en œuvre des mécanismes
          d'auto-centrage et d'auto-équilibrage.
    \item \href{https://www.ctan.org/pkg/minted}{minted} Le paquetage qui facilite la coloration syntaxique expressive dans
          \gls{latex} en utilisant la puissante bibliothèque Pygments
    \item \href{https://www.ctan.org/pkg/caption}{caption} Le paquetage de légendes offre de nombreuses façons de
          personnaliser les légendes dans les environnements flottants comme les figures et les tableaux, et coopère
          avec de nombreux autres paquets.
    \item \href{https://www.ctan.org/pkg/biblatex}{biblatex} BibLaTeX est une réimplémentation complète des fonctions
          bibliographiques fournies par \gls{latex}. BibLaTeX utilise son propre programme de gestion des données appelé biber
          pour lire et traiter les données bibliographiques. Grâce à biber, BibLaTeX dispose de nombreuses fonctionnalités
          rivalisant avec les autres systèmes bibliographiques, voire les surpassant.
\end{itemize}

\subsubsection{Dictionnaire} \label{dictionnaire}
Il est possible de créer un dictionnaire qui pourra être réutilisé dans le document.
Pour cela, il faut renseigner ses définitions dans le fichier \verb=glossary.tex= de la manière suivante :
\begin{code}
    \begin{minted}{latex}
        \newglossaryentry{la clé pour appeller}
        {
        name={Le mot à définir},
        description={La définition},
        first={La définition (Le mot)},
        text={Le mot},
        }

        % Par exemple
        \newglossaryentry{ed}
        {
            name={Exemple Dictionnaire},
            description={Terme définissant le mot},
            first={Exemple Dictionnaire (ED)},
            text={ED},
        }
\end{minted}
    \caption{Définition dans le dictionnaire}
\end{code}

Si un mot est utilisé plusieurs fois dans le document et que l'orthographe de ce mot doit être modifié;
cela peut se faire facilement et dans tout le document en le modifiant dans \verb=glossary.tex=.
On utilise, pour appeler ce mot, la même manière que pour les acronymes, \ref{acronyme} page \pageref{acronyme}, :
\begin{code}
    \begin{minted}{latex}
        \gls{ed}
\end{minted}
    \caption{Utilisation du dictionnaire}
\end{code}

Si c'est la première fois que le mot est appelé c'est la définition dans l'option \verb=first= qui apparaît : \gls{ed}.
Une note de bas de page apparaît aussi avec la définition. \newline
Si ça n'est pas la première fois: \gls{ed}. C'est le mot dans l'option \verb=text= qui apparaît et il n'y a pas de note de bas de page.

Il est possible de faire apparaitre la liste des définitions. Ici c'est en fin de document,
page \pageref{dictionnaireDomaine}, trié par ordre alphabétique:

\subsubsection{Acronyme} \label{acronyme}
Il est possible de créer une liste d'acronymes qui pourra être réutilisée dans le document.
Pour cela, il faut renseigner ses acronymes dans le fichier \verb=acronyms.tex= de la manière suivante :
\begin{code}
    \begin{minted}{latex}
        \newacronym[first=Définition de l'acronyme (l'acronyme)]{la clé pour appeller}{l'acronyme}{Définition de l'acronyme}

        % Par exemple
        \newacronym[first=Définition de l'acronyme (ARC)]{acr}{ARC}{Définition de l'acronyme}
\end{minted}
    \caption{Définition d'un acronyme}
\end{code}

Si un acronyme est utilisé plusieurs fois dans le document et que l'orthographe de cet acronyme doit être modifiée;
cela peut se faire facilement et dans tout le document en le modifiant dans \verb=acronyms.tex=.
On utilise, pour appeler cet acronyme, la même manière que pour le dictionnaire, \ref{dictionnaire} page \pageref{dictionnaire}, :
\begin{code}
    \begin{minted}{latex}
        \gls{acr}
\end{minted}
    \caption{Utilisation d'un acronyme}
\end{code}

Si c'est la première fois que l'acronyme est appelé c'est la définition dans l'option \verb=first= qui apparaît : \gls{acr}.
Une note de bas de page apparaît aussi. \newline
Si ça n'est pas la première fois: \gls{acr}. C'est l'acronyme qui apparaît et il n'y a pas de note de bas de page. \newline

Il est possible de faire apparaitre la liste des acronymes défini. Ici c'est dans un tableau trié par ordre alphabétique:

\newpage
\setglossarysection{paragraph} % Type de titre
\printglossary[type=\acronymtype,style=superheaderborder ,title={Définitions, acronymes et abréviations}]
\glsaddallunused % Même les acronymes non utilisés sont affichés

\subsubsection{PlantUML}
Pour la génération des PlantUMLs en images il est nécessaire d'avoir installé:
\begin{itemize}
    \item Java, utilisation de \verb=plantuml.jar=. Le chemin relatif vers
          \verb=plantuml.jar= doit être indiqué dans le \verb=makefile= via la variable \verb=PLANTUML=.
    \item Graphviz, génère un graphe à partir d'une description textuelle de PlantUML
\end{itemize}

Pour générer l'ensemble des diagrammes il faut faire la commande :
\begin{code}
    \begin{minted}{bash}
    make eps
\end{minted}
    \caption{Génération des diagrammes}
\end{code}

Pour supprimer tout les fichiers générés par les diagrammes :
\begin{code}
    \begin{minted}{bash}
    make clean_eps
\end{minted}
    \caption{Nettoyage des diagrammes}
\end{code}

La commande va récupérer l'ensemble des fichiers \verb=.plantuml=
dans le dossier \verb=schémas/= pour les convertir en \verb=.eps=.
Il s'agit d'un format d'image vectorielle supporté par \gls{latex}.
Garantissant une grande qualité d'image quelle que soit la taille.

Les fichier sont de la forme suivant :
\begin{code}
    \begin{minted}{vim}
        @startuml
        skinparam shadowing false

        Alice -> Bob : hello

        @enduml ' Fin du document
    \end{minted}
    \caption{Exemple de fichier PlantUMLs}
\end{code}

\begin{figure}[H]
    \centering
    \includegraphics[width=2cm]{exemple}
    \caption{Exemple de diagramme UML}
\end{figure}

Il est posible de ranger les diagrammes dans des sous-dossiers dont il faudra préciser le chemin relatif.
\begin{figure}[H]
    \centering
    \includegraphics[width=2cm]{exemple/exempleFolder}
    \caption{Exemple de diagramme UML dans un sous dossier }
\end{figure}

\subsubsection{Minted}
Il est possible de mettre des extraits de code dans le document via le paquetage Minted.
Ce paquetage utilise la librairie Python \verb=Pygments=.
Cette librairie doit être installée avec le gestionnaire de paquets \verb=pip= qui doit lui-même être installé.
\begin{code}
    \begin{minted}{bash}
    pip install pygments
\end{minted}
    \caption{Commande d'installation de pygments}
\end{code}

Les extraits de code seront colorisés en fonction du langage mit en paramètre comme ci-dessous :

\begin{code}
    \begin{minted}{c}
    for(int i = 0; i < 10; i++) {
        println("Hello World\n");
    }
\end{minted}
    \caption{Hello World en C}
\end{code}

\begin{code}
    \begin{minted}{java}
    class HelloWorld {
        public static void main(String[] args) {
            System.out.println("Hello, World!");
        }
    }
\end{minted}
    \caption{Hello World en Java}
\end{code}

Les extraits de code sont mis dans un environnement crée pour l'occasion : \verb=code=.
Le but étant d'ajouter des légendes et une mise en page particulière et de les lister par la suite à la fin du document,
page \pageref{TableOfCode}.

\subsubsection{Xltabular}
Il est possible de créer des tableaux pour qu'ils puissent s'étaler sur l'ensemble de la page et sur plusieurs pages.
L'objectif de placer le tableau sur plusieurs page sans qu'une ligne ne soit divisée en deux.
Voici un exemple pour un tableau à 5 colonnes:
\begin{code}
    \begin{minted}{latex}
        \begin{xltabular}{\linewidth}{|l|X|r|c|p{0.15\linewidth}|}
            % l = left, X = place restante et retour à la ligne, r = right, c = center, p = place
            % Si aucune taille n'est donnée aux colonnes, elle se fait en fonction du texte dans la colonne

            % Titre première page
            \hline \textbf{Gauche} & \textbf{Retour à la ligne} & \textbf{Droite} & \textbf{Centre} & \textbf{Taille fixe}\\\hline
            \endfirsthead

            \multicolumn{5}{l}% Fusion des colones et texte à gauche
            {\textbf{\dots\space suite de la page précédente}}\\
            % Titre seconde page et plus
            \hline \textbf{Gauche} & \textbf{Retour à la ligne} & \textbf{Droite} & \textbf{Centre} & \textbf{Taille fixe}\\\hline
            \endhead

            \multicolumn{5}{r}% Fusion des colones et texte à droite
            {\textbf{Suite à la page suivante\dots}}\tabularnewline
            \endfoot
            \endlastfoot

            Gauche & Retour à la ligne & Droite & Centre & Taille fixe\\ \hline
        \end{xltabular}
\end{minted}
    \caption{Exemple tableau}
\end{code}

Ce qui donnera le résultat :
\begin{xltabular}{\linewidth}{|l|X|r|c|p{0.15\linewidth}|}
    % l = left, X = place restante et retour à la ligne, r = right, c = center, p = place
    % Si aucune taille n'est donnée aux colonnes, elle se fait en fonction du texte dans la colonne

    % Titre première page
    \hline \textbf{Gauche} & \textbf{Retour à la ligne} & \textbf{Droite} & \textbf{Centre} & \textbf{Taille fixe}\\\hline
    \endfirsthead

    \multicolumn{5}{l}% Fusion des colones et texte à gauche
    {\textbf{\dots\space suite de la page précédente}}\\
    % Titre seconde page et plus
    \hline \textbf{Gauche} & \textbf{Retour à la ligne} & \textbf{Droite} & \textbf{Centre} & \textbf{Taille fixe}\\\hline
    \endhead

    \multicolumn{5}{r}% Fusion des colones et texte à droite
    {\textbf{Suite à la page suivante\dots}}\tabularnewline
    \endfoot
    \endlastfoot

    Gauche & Retour à la ligne & Droite & Centre & Taille fixe\\ \hline
    Gauche & \lipsum[1] & Droite & Centre & Taille fixe\\ \hline
    Gauche & \lipsum[3] & Droite & Centre & Taille fixe\\ \hline
    Gauche & \lipsum[5] & Droite & Centre & Taille fixe\\ \hline
\end{xltabular}

Si aucune taille n'est donnée dans les colones \verb=Gauche Droite Centre=, la largueur de la colonne
se fera en fonction du texte qui est mit dans la colonne. On riques d'avoir des tableaux trop large qui dépassent de la page
comme ici :
\begin{xltabular}{\linewidth}{|l|X|r|c|p{0.15\linewidth}|}
    % l = left, X = place restante et retour à la ligne, r = right, c = center, p = place
    % Si aucune taille n'est donnée aux colonnes, elle se fait en fonction du texte dans la colonne

    % Titre première page
    \hline \textbf{Gauche} & \textbf{Retour à la ligne} & \textbf{Droite} & \textbf{Centre} & \textbf{Taille fixe}\\\hline
    \endfirsthead

    \multicolumn{5}{l}% Fusion des colones et texte à gauche
    {\textbf{\dots\space suite de la page précédente}}\\
    % Titre seconde page et plus
    \hline \textbf{Gauche} & \textbf{Retour à la ligne} & \textbf{Droite} & \textbf{Centre} & \textbf{Taille fixe}\\\hline
    \endhead

    \multicolumn{5}{r}% Fusion des colones et texte à droite
    {\textbf{Suite à la page suivante\dots}}\tabularnewline
    \endfoot
    \endlastfoot

    \lipsum[1] & Retour à la ligne & Droite & Centre & Taille fixe\\ \hline
\end{xltabular}

Sauf dans certain cas, voir page \pageref{TableOfVersion}, il est préférable de tout fixer comme ici :
\begin{code}
    \begin{minted}{latex}
        \begin{xltabular}{\linewidth}{|p{0.6\linewidth}|X|}
            % La colonne fait 0,6 fois largueur d'une ligne
            \hline \textbf{Colonne 1} & \textbf{Colonne 2} \\\hline
            \endfirsthead

            \multicolumn{2}{l}%
            {\textbf{\dots\space suite de la page précédente}}\\
            \hline \textbf{Colonne 1} & \textbf{Colonne 2} \\\hline
            \endhead

            \multicolumn{2}{r}% Fusion des colones et texte à droite
            {\textbf{Suite à la page suivante\dots}}\tabularnewline
            \endfoot
            \endlastfoot

            \lipsum[1] & \lipsum[2] \\ \hline
        \end{xltabular}
\end{minted}
    \caption{Exemple tableau taille fixe}
\end{code}

\begin{xltabular}{\linewidth}{|p{0.6\linewidth}|X|}
    % La colonne fait 0,6 fois largueur d'une ligne
    \hline \textbf{Colonne 1} & \textbf{Colonne 2} \\\hline
    \endfirsthead

    \multicolumn{2}{l}%
    {\textbf{\dots\space suite de la page précédente}}\\
    \hline \textbf{Colonne 1} & \textbf{Colonne 2} \\\hline
    \endhead

    \multicolumn{2}{r}% Fusion des colones et texte à droite
    {\textbf{Suite à la page suivante\dots}}\tabularnewline
    \endfoot
    \endlastfoot

    \lipsum[1] & \lipsum[2] \\ \hline
\end{xltabular}

Ainsi, on évite des débordements. À noter que si une colonne n'est pas assez large et que le texte à l'intérieur fait
plus d'une longueur de page cela déborde aussi.

\subsubsection{Biblatex}
Il est possible de créer une bibliographie qui pourra être réutilisée dans le document.
Pour cela, il faut renseigner ses références dans le fichier \verb=biblio.bib= de la manière suivante :
\begin{code}
    \begin{minted}{latex}
        @article{la clé pour appeller,
            author  = {auteur},
            title   = {Titre},
            ...
        }

        % Par exemple
        @article{CitekeyArticle,
            author  = {P. J. Cohen},
            title   = {The independence of the continuum hypothesis},
            journal = {Proceedings of the National Academy of Sciences},
            year    = 1963,
            volume  = {50},
            number  = {6},
            pages   = {1143--1148}
        }
\end{minted}
    \caption{Définition dans la bibliographie}
\end{code}

Si une référence, est utilisée plusieurs fois dans le document et que l’orthographe de cette référence
doit être modifié ; cela peut se faire facilement et dans tout le document en le modifiant dans
\verb=biblio.bib=.
Pour appeler cette référence peut utiliser différentes manières:
\begin{code}
    \begin{minted}{latex}
        \autocite{CitekeyArticle} % Ajoute la référence en note de bas de page
        \cite{CitekeyArticle}     % Ajoute la référence à la suite
\end{minted}
    \caption{Utilisation de la bibliographie}
\end{code}

Ce qui donne le résultat suivant où l'on ajoute la note de bas de page \autocite{CitekeyArticle}.
Et ici on a la fait apparaitre : \cite{CitekeyArticle}. \newline

Dans notre exemple on affiche le nom de l'auteur et le titre. Il existe d'autres manière où l'on affiche d'autres informations.
Voir la \href{https://mirror.ibcp.fr/pub/CTAN/macros/latex/contrib/biblatex/doc/biblatex.pdf#subsection.3.9}{documantation CTAN}. \newline

Dans le fichier \verb=biblio.bib= se trouve, en exemple, l'ensemble des catégories de référence qui existe.
En fonction de la catégorie, les informations demandées sont différentes et leurs affichage le sont aussi,
voir page \pageref{bibliographie}.